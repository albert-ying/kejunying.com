%!TEX TS-program = xelatex
\documentclass{nihbiosketch}

% URL settings for better line breaks
\PassOptionsToPackage{hyphens,spaces,obeyspaces}{url}
\PassOptionsToPackage{breaklinks=true,hidelinks,unicode}{hyperref}

% Make URLs more compact in bibliography
\makeatletter
\def\doi#1{\href{https://doi.org/#1}{\small doi:#1}}
\g@addto@macro{\UrlBreaks}{\UrlOrds}
\makeatother

% Adjust paragraph settings for better line breaks
\setlength{\emergencystretch}{3em}
\tolerance=1000

%------------------------------------------------------------------------------

\name{BAKER, DAVID}
\eracommons{DABAKER}
\position{Director}

\begin{document}
%------------------------------------------------------------------------------

\begin{education}
Harvard University, Cambridge, MA             & B.A.          & 1984  & Biology \\
Univ of California, Berkeley, CA             & Ph.D.         & 12/1989  & Biochemistry \\
Univ of California, San Francisco, CA        & Postdoctoral Fellow & 1993  & Biophysics \\
\end{education}


\section{Personal Statement}

\begin{statement}

My colleagues and I have developed the Rosetta computational methodology for predicting and designing macromolecular structures, interactions, and functions. We have used this methodology to predict the structures and interactions of naturally occurring biomolecules, and to design new proteins with new structures, interactions and functions. More recently, we have been developing deep learning software for protein design which is enabling highly accurate and robust design of new protein binders, catalysts and self-assembling materials.

\end{statement}

%------------------------------------------------------------------------------
\section{Positions, Scientific Appointments, and Honors}

\subsection*{Positions and Scientific Appointments}
\begin{datetbl}
2012--Present & Investigator, Howard Hughes Medical Institute \\
2005--Present & Director, Institute for Protein Design, University of Washington, Seattle, WA \\
2004--Present & Professor, University of Washington, Department of Biochemistry, Seattle, WA \\
2004--Present & Adjunct Professor of Bioengineering, Computer Science, Genome Sciences, Chemical Engineering, and Physics, University of Washington, Seattle, WA \\
2000--2004    & Associate Investigator, Howard Hughes Medical Institute \\
2000--2005    & Associate Professor, University of Washington, Department of Biochemistry, Seattle, WA \\
1993--2000    & Assistant Professor, University of Washington, Department of Biochemistry, Seattle, WA \\
\end{datetbl}

\subsection*{Honors}
\begin{datetbl}
2024          & Nobel Prize in Chemistry, Royal Swedish Academy of Sciences \\
2021          & Prize in Life Sciences, Breakthrough \\
2020          & Fellow, American Institute for Medical and Biological Engineering \\
2018          & Public Lecture, International Solvay Institutes \\
2018          & Hans Neurath Award, Protein Society \\
2017          & Henrietta and Aubrey Davis Endowed Professorship in Biochemistry, University of Washington \\
2014          & Perlman Memorial Award, American Chemical Society \\
2012          & Centenary Award, Biochemical Society \\
2011          & Inventor of the Year Award, University of Washington \\
2009          & Member, American Academy of Sciences \\
2008          & Sackler Prize in Biophysics, Tel Aviv University \\
2007          & Editorial Board, Proceedings of the National Academy of Sciences \\
2006          & Member, National Academy of Sciences \\
2004          & Feynman Prize, Foresight Institute \\
2004          & Newcomb Cleveland Prize, American Academy of Sciences \\
2002          & Overton Prize, International Society for Computational Biology \\
2000          & Young Investigator Award, Beckman \\
1994          & Fellowship in Science and Engineering, Packard \\
1994          & Young Investigator Award, National Science Foundation \\
1993          & Fellowship, Boyer Foundation \\
\end{datetbl}

%------------------------------------------------------------------------------

\section{Contribution to Science}

\begin{enumerate}

\item \textbf{How proteins fold.} To investigate the extent to which amino acid sequences determine protein folding rates, we used selection methods to identify very different sequences that fold to the same structures. While the stabilities of these proteins were generally less than the native protein, the folding rates were as often faster as slower, suggesting that evolution has not optimized sequences for rapid folding. Since introducing substantial sequence variation did not significantly affect protein folding rates, we sought other factors that determined the rate of folding. A series of experimental and computational analyses established that protein folding rates and mechanisms are largely determined by the topology of the native structure. A particularly notable observation was that protein folding rates are strongly correlated with the contact order of the native structure (the sequence separation between contacting residues in the protein structure), with low contact order proteins folding orders of magnitude faster than high contact order proteins.

\begin{enumerate}

\item Rocklin GJ, Chidyausiku TM, Goreshnik I, Ford A, Houliston S, Lemak A, Carter L, Ravichandran R, Mulligan VK, Chevalier A, Arrowsmith CH, Baker D. Global analysis of protein folding using massively parallel design, synthesis, and testing. \textit{Science}. 2017 Jul 14;357(6347):168-175. PubMed Central PMCID: PMC5568797.

\item Riddle DS, Grantcharova VP, Santiago JV, Alm E, Ruczinski I, Baker D. Experiment and theory highlight role of native state topology in SH3 folding. \textit{Nat Struct Biol}. 1999 Nov;6(11):1016-24. PubMed PMID: 10542092.

\item Alm E, Baker D. Prediction of protein-folding mechanisms from free-energy landscapes derived from native structures. \textit{Proc Natl Acad Sci U S A}. 1999 Sep 28;96(20):11305-10. PubMed Central PMCID: PMC18029.

\item Plaxco KW, Simons KT, Baker D. Contact order, transition state placement and the refolding rates of single domain proteins. \textit{J Mol Biol}. 1998 Apr 10;277(4):985-94. PubMed PMID: 9545386.

\end{enumerate}


\item \textbf{Protein structure prediction and deep learning.} Guided by the insights gained in our studies of protein folding mechanism, we developed the Rosetta ab initio structure prediction methodology which builds protein structures by fragment assembly. The CASP blind structure prediction experiments showed that the Rosetta protein structure prediction methodology was a significant improvement over previous approaches. We developed methods for efficiently refining protein models in a physically realistic all atom potential, coupled this with lower resolution conformational search methods, and showed that not only monomeric protein structures but also protein-protein complexes (the docking problem), higher order symmetric protein assemblies, membrane proteins, and RNA structures could be modeled accurately by searching for the lowest energy state provided the space to be searched was not too large. We then showed that the Rosetta approach could generate quite accurate models of more complex systems when provided with limited experimental data to guide conformational sampling. Rosetta supplemented with experimental data has become a powerful and widely used approach to solve macromolecular structures using sparse NMR data (CS-Rosetta), low-resolution x-ray diffraction data (MR-Rosetta), cryo-electron microscopy data, and co-evolution sequence information (Gremlin-Rosetta).

\begin{enumerate}

\item Ovchinnikov S, Park H, Varghese N, Huang PS, Pavlopoulos GA, Kim DE, Kamisetty H, Kyrpides NC, Baker D. Protein structure determination using metagenome sequence data. \textit{Science}. 2017 Jan 20;355(6322):294-298. PubMed Central PMCID: PMC5493203.

\item DiMaio F, Terwilliger TC, Read RJ, Wlodawer A, Oberdorfer G, Wagner U, Valkov E, Alon A, Fass D, Axelrod HL, Das D, Vorobiev SM, Iwaï H, Pokkuluri PR, Baker D. Improved molecular replacement by density- and energy-guided protein structure optimization. \textit{Nature}. 2011 May 26;473(7348):540-3. PubMed Central PMCID: PMC3365536.

\item Raman S, Lange OF, Rossi P, Tyka M, Wang X, Aramini J, Liu G, Ramelot TA, Eletsky A, Szyperski T, Kennedy MA, Prestegard J, Montelione GT, Baker D. NMR structure determination for larger proteins using backbone-only data. \textit{Science}. 2010 Feb 19;327(5968):1014-8. PubMed Central PMCID: PMC2909653.

\item Gray JJ, Moughon SE, Kortemme T, Schueler-Furman O, Misura KM, Morozov AV, Baker D. Protein-protein docking predictions for the CAPRI experiment. \textit{Proteins}. 2003 Jul 1;52(1):118-22. PubMed PMID: 12784377; NIHMSID: NIHMS308621.

\end{enumerate}


\item \textbf{Design of protein structure and immunogens.} The Rosetta structure prediction methodology described above searches for the lowest energy structure for a given sequence, and we realized that we could invert the process to search for the lowest energy sequence for a desired structure—the protein design problem. We demonstrated proof-of-concept for de novo protein design with the design of TOP7, a novel protein with a fold not found in nature, and later developed general principles for designing hyperstable idealized alpha-beta proteins and helical bundles. With the capability of designing stable protein structures in hand, we developed methods for stabilizing both linear and complex epitopes from pathogen proteins, and showed that these could elicit neutralizing antibodies in animals, opening up computational design approaches to developing improved vaccines. More recently, we have developed and been using a guided diffusion model called RFdiffusion to design even more complex structures and functions.

\begin{enumerate}

\item Hosseinzadeh P, Bhardwaj G, Mulligan VK, Shortridge MD, Craven TW, Pardo-Avila F, Rettie SA, Kim DE, Silva DA, Ibrahim YM, Webb IK, Cort JR, Adkins JN, Varani G, Baker D. Comprehensive computational design of ordered peptide macrocycles. \textit{Science}. 2017 Dec 15;358(6369):1461-1466. PubMed Central PMCID: PMC5860875.

\item Correia BE, Bates JT, Loomis RJ, Baneyx G, Carrico C, Jardine JG, Rupert P, Correnti C, Kalyuzhniy O, Vittal V, Connell MJ, Stevens E, Schroeter A, Chen M, Macpherson S, Serra AM, Adachi Y, Holmes MA, Li Y, Klevit RE, Graham BS, Wyatt RT, Baker D, Strong RK, Crowe JE Jr, Johnson PR, Schief WR. Proof of principle for epitope-focused vaccine design. \textit{Nature}. 2014 Mar 13;507(7491):201-6. PubMed Central PMCID: PMC4260937.

\item Koga N, Tatsumi-Koga R, Liu G, Xiao R, Acton TB, Montelione GT, Baker D. Principles for designing ideal protein structures. \textit{Nature}. 2012 Nov 8;491(7423):222-7. PubMed Central PMCID: PMC3705962.

\item Kuhlman B, Dantas G, Ireton GC, Varani G, Stoddard BL, Baker D. Design of a novel globular protein fold with atomic-level accuracy. \textit{Science}. 2003 Nov 21;302(5649):1364-8. PubMed PMID: 14631033.

\end{enumerate}


\item \textbf{Design of small molecule binding, catalysis, protein interactions, and self-assembly.} A grand challenge in computational protein design is creating new binding proteins de novo for use in therapeutics and diagnostics. We have developed general methods for designing proteins which bind with high affinity/specificity to sites of interest of therapeutic importance on protein targets, both human and pathogen (influenza, bacterial toxins, oncogenic proteins). We developed general methods for designing catalysts for arbitrary chemical reactions starting from a description of the reaction transition state geometry, and used the approach to design catalysts for a number of reactions not catalyzed by naturally occurring enzymes. We have developed an approach to computationally designing self-assembling nanomaterials and used it to design new proteins that self-assemble into regular tetrahedral, octahedral and icosahedral structures as well as two dimensional layers with near atomic-level accuracy. We are now developing these self-assembling materials for targeted delivery and vaccine applications.

\begin{enumerate}

\item Yeh AH, Norn C, Kipnis Y, Tischer D, Pellock SJ, Evans D, Ma P, Lee GR, Zhang JZ, Anishchenko I, Coventry B, Cao L, Dauparas J, Halabiya S, DeWitt M, Carter L, Houk KN, Baker D. De novo design of luciferases using deep learning. \textit{Nature}. 2023 Feb;614(7949):774-780. PubMed Central PMCID: PMC9946828.

\item Chevalier A, Silva DA, Rocklin GJ, Hicks DR, Vergara R, Murapa P, Bernard SM, Zhang L, Lam KH, Yao G, Bahl CD, Miyashita SI, Goreshnik I, Fuller JT, Koday MT, Jenkins CM, Colvin T, Carter L, Bohn A, Bryan CM, Fernández-Velasco DA, Stewart L, Dong M, Huang X, Jin R, Wilson IA, Fuller DH, Baker D. Massively parallel de novo protein design for targeted therapeutics. \textit{Nature}. 2017 Oct 5;550(7674):74-79. PubMed Central PMCID: PMC5802399.

\item King NP, Bale JB, Sheffler W, McNamara DE, Gonen S, Gonen T, Yeates TO, Baker D. Accurate design of co-assembling multi-component protein nanomaterials. \textit{Nature}. 2014 Jun 5;510(7503):103-8. PubMed Central PMCID: PMC4137318.

\item Tinberg CE, Khare SD, Dou J, Doyle L, Nelson JW, Schena A, Jankowski W, Kalodimos CG, Johnsson K, Stoddard BL, Baker D. Computational design of ligand-binding proteins with high affinity and selectivity. \textit{Nature}. 2013 Sep 12;501(7466):212-216. PubMed Central PMCID: PMC3898436.

\end{enumerate}


\item \textbf{Involving the general public in Science.} We created a distributed computing project called Rosetta@home in which volunteers donate spare cycles on their computers to carry out protein folding trajectories. We extended Rosetta@home to the interactive multiplayer online game Foldit which allows players to guide the course of the protein structure prediction and design calculations. By relying on human intuition and 3-D problem solving skills, Foldit players have made a number of important contributions: solved the structure a retroviral protease, developed new algorithms for finding low-energy protein conformations, and designed a novel synthetic enzyme by large-scale redesign of the active site.

\begin{enumerate}

\item Koepnick B, Flatten J, Husain T, Ford A, Silva DA, Bick MJ, Bauer A, Liu G, Ishida Y, Boykov A, Estep RD, Kleinfelter S, Nørgård-Solano T, Wei L, Players F, Montelione GT, DiMaio F, Popović Z, Khatib F, Cooper S, Baker D. De novo protein design by citizen scientists. \textit{Nature}. 2019 Jun;570(7761):390-394. PubMed Central PMCID: PMC6701466.

\item Eiben CB, Siegel JB, Bale JB, Cooper S, Khatib F, Shen BW, Players F, Stoddard BL, Popovic Z, Baker D. Increased Diels-Alderase activity through backbone remodeling guided by Foldit players. \textit{Nat Biotechnol}. 2012 Jan 22;30(2):190-2. PubMed Central PMCID: PMC3566767.

\item Khatib F, Cooper S, Tyka MD, Xu K, Makedon I, Popovic Z, Baker D, Players F. Algorithm discovery by protein folding game players. \textit{Proc Natl Acad Sci U S A}. 2011 Nov 22;108(47):18949-53. PubMed Central PMCID: PMC3223433.

\item Khatib F, DiMaio F, Cooper S, Kazmierczyk M, Gilski M, Krzywda S, Zabranska H, Pichova I, Thompson J, Popović Z, Jaskolski M, Baker D. Crystal structure of a monomeric retroviral protease solved by protein folding game players. \textit{Nat Struct Mol Biol}. 2011 Sep 18;18(10):1175-7. PubMed Central PMCID: PMC3705907.

\end{enumerate}

\end{enumerate}

\subsection*{Complete List of Published Work:} 
\url{https://www.ncbi.nlm.nih.gov/myncbi/david.baker.1/bibliography/public/}

\end{document}
